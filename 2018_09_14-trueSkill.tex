\documentclass[shownotes]{beamer}

\mode<presentation>
{
%   \usetheme{Madrid}      % or try Darmstadt, Madrid, Warsaw, ...
%   \usecolortheme{default} % or try albatross, beaver, crane, ...
%   \usefonttheme{default}  % or try serif, structurebold, ...
 \usetheme{Antibes}
 \usecolortheme[rgb={0.6,0.75,0}]{structure}%divido los RGB por 252
 \setbeamercolor{block title}{fg=white,bg=azuluca}
 \xdefinecolor{azuluca}{rgb}{0.02, 0.2, 0.18}
 \definecolor{greenblue}{rgb}{0.1, 0.55, 0.5}

 \setbeamercolor{palette quaternary}{fg=white,bg=azuluca}
 \setbeamertemplate{caption}[numbered]
 \setbeamertemplate{navigation symbols}{}

}

\newcommand\hfrac[2]{\genfrac{}{}{0pt}{}{#1}{#2}}

\newtheorem{midef}{Definition}
\newtheorem{miteo}{Theorem}
\newtheorem{mipropo}{Proposition}

\usepackage[utf8]{inputenc} %Para acentos en UTF8 (Prueba: á é í ó ú Á É Í Ó Ú ñ Ñ)
\usepackage[spanish]{babel}
\usepackage{url}
%\usepackage{mathtools}
\usepackage{graphicx}
\usepackage{caption}
\usepackage{float} % para que los gr\'aficos se queden en su lugar con [H]
\usepackage{subcaption}
\usepackage{wrapfig}
\usepackage{color}
\usepackage{amsmath} %para escribir funci\'on partida
\usepackage{blkarray}
\usepackage{hyperref} % para inlcuir links dentro del texto
\usepackage{mdframed} 
\usepackage{comment}
\usepackage{amsfonts} % mathbb{N} -> conjunto de los n\'umeros naturales  
\usepackage{enumerate}
\usepackage{listings}
\usepackage{enumitem}
\usepackage{framed}
\usepackage{mdframed}
\usepackage{bm}
\usepackage{transparent}

\usepackage[absolute,overlay]{textpos} %no funciona
\setlength{\TPHorizModule}{1mm} %128mm  mitad: 64 
\setlength{\TPVertModule}{1mm}	%96mm  mitad 48

\setlist[enumerate,1]{start=0} % only outer nesting level

\usepackage{tikz} % Para graficar, por ejemplo bayes networks
\usetikzlibrary{bayesnet} % Para que ande se necesita copiar el archivo  tikzlibrarybayesnet.code.tex en la misma carpeta

%%%%%%%%%%%%%%%%%%%%%%%%%%%%%%%%%5
%
% Incompatibles con textpos
%
%\usepackage{todonotes}
%\usepackage{tikz} % Para graficar, por ejemplo bayes networks
%
%%%%%%%%%%%%%%%%%%%%%%%%%%%%%%%%%%
 
\usepackage[absolute,overlay]{textpos} %no funciona
\setlength{\TPHorizModule}{1mm} %128mm  mitad: 64 
\setlength{\TPVertModule}{1mm}	%96mm  mitad 48
% 
% 
\captionsetup[figure]{labelformat=empty}

\hypersetup{
    colorlinks=true,       % false: boxed links; true: colored links
    linkcolor=white,          % color of internal links (change box color with linkbordercolor)
    citecolor=green,        % color of links to bibliography
    filecolor=magenta,      % color of file links
    urlcolor=cyan           % color of external links
}
% 
% http://latexcolor.com/
\definecolor{lightseagreen}{rgb}{0.13, 0.7, 0.6.5}
\definecolor{greenblue}{rgb}{0.1, 0.55, 0.5}
\definecolor{redgreen}{rgb}{0.6, 0.4, 0.}
\definecolor{greenred}{rgb}{0.4, 0.7, 0.}
\definecolor{redblue}{rgb}{0.4, 0., .4}
\definecolor{tangelo}{rgb}{0.98, 0.3, 0.0}
\definecolor{git}{rgb}{0.94, 0.309, 0.2}
% 
\setbeamercolor{structure}{fg=greenblue}

\newcommand\Wider[2][3em]{%
\makebox[\linewidth][c]{%
  \begin{minipage}{\dimexpr\textwidth+#1\relax}
  \raggedright#2
  \end{minipage}%
  }%
}


\usefonttheme[onlymath]{serif}

\title[Estimaci\'on de habilidad]{Toda la matem\'atica detr\'as de TrueSkill}

\author[Gustavo Landfried]{Gustavo Landfried \\ \vspace{0.2cm}
\scriptsize Lic. Ciencias Antropol\'ogicas. \\
Doctorando Ciencias de la Computaci\'on \\
\vspace{-0.3cm}}
\institute[DC-ICC-CONICET]{Instituto de Ciencias de la Computaci\'on (UBA -- CONICET) \vspace{-0.3cm}}
\date{}

\begin{document}

\begin{frame}[noframenumbering]
 \vspace{-0.5cm}
  \begin{figure}[H]     
     \centering
     \begin{subfigure}[b]{0.50\textwidth}
       \includegraphics[page=1,width=\textwidth]{static/manuscript.png} 
     \end{subfigure}
   \end{figure} 

\vspace{-1cm}
\maketitle

 \begin{textblock}{80}(72,82)
 \includegraphics[width=0.2\textwidth]{images/logo_licar} 
 \end{textblock}
  \begin{textblock}{80}(42,82)
 \includegraphics[width=0.2\textwidth]{images/logo_version_02} 
\end{textblock}

\end{frame}
\small

\section{Problema}

\begin{frame}

\begin{mdframed}[backgroundcolor=black!15]
 \centering \Large
 ¿C\'omo estimar la habilidad de las personas?
\end{mdframed}

\vspace{0.25cm}

\begin{center}
 \large Depende del tipo de tarea
\end{center}

\begin{enumerate}
 \item Tarea fija
 \item Relativa a otros
\end{enumerate}

  \begin{figure}[H]     
     \centering
     \begin{subfigure}[b]{0.25\textwidth}
       \includegraphics[page=1,width=\textwidth]{static/persona_caja.jpeg} 
     \caption*{Tarea fija}
     \end{subfigure}
     \hspace{1cm}
     \begin{subfigure}[b]{0.25\textwidth}
       \includegraphics[page=2,width=\textwidth]{static/persona_persona.jpeg} 
       \caption*{Relativa a otros}
     \end{subfigure}
   \end{figure} 
\end{frame}


\begin{frame}[noframenumbering]

\begin{mdframed}[backgroundcolor=black!15]
 \centering \Large
 ¿C\'omo estimar la habilidad de las personas?
\end{mdframed}

\vspace{0.25cm}

\begin{center}
 \large Depende del tipo de tarea
\end{center}

\begin{enumerate}
 \item Tarea fija
 \item \textbf{Relativa a otros}
\end{enumerate}

  \begin{figure}[H]     
     \centering
     \begin{subfigure}[b]{0.25\textwidth}
       \includegraphics[page=1,width=\textwidth]{static/persona_caja.jpeg} 
     \caption*{Tarea fija}
     \end{subfigure}
     \hspace{1cm}
     \begin{subfigure}[b]{0.25\textwidth}
       \includegraphics[page=2,width=\textwidth]{static/persona_persona.jpeg} 
       \caption*{\textbf{Relativa a otros}}
     \end{subfigure}
   \end{figure} 
\end{frame}

\section{Elo}

\begin{frame}

\begin{center}
 \large Sistema de puntuaci\'on Elo
\end{center}


  \begin{figure}[H]     
     \centering
     \begin{subfigure}[b]{0.4\textwidth}
       \includegraphics[page=1,width=\textwidth]{images/elo.jpeg} 
       \caption*{Arpad Elo}
     \end{subfigure}
\end{figure}


\end{frame}

\subsection{Modelo}

\begin{frame}

\begin{center}
 \Large Modelo
\end{center}


\begin{equation}
 \large p \sim N(s,\beta)
\end{equation}

\begin{itemize} \normalsize
 \item[$p$)]  \textbf{Rendimiento} o \emph{performance}
 \item[$s$)]  \textbf{Habilidad} o \emph{skill}
 \item[$\beta$)] Ruido 
\end{itemize}

\begin{equation}
 \large r_{ij} = \mathbb{I}_{\{p_i > p_j \}} = \mathbb{I}_{\{d_{ij} > 0 \}}
\end{equation}

\begin{itemize} \normalsize
 \item[$r$)]  \textbf{Resultado}
 \item[$d$)]  Diferencia $d_{ij} = p_i - p_j$
\end{itemize}


\end{frame}



\subsection{Probabilidad de ganar}


\begin{frame}

\begin{center}
 \large Probabilidad de ganar (gr\'afica)
\end{center}
\vspace{-0.3cm}
\begin{figure}[H]     
     \centering
     \begin{subfigure}[b]{0.7\textwidth}
       \includegraphics[page=1,width=\textwidth]{figures/probaOfWin_2D} 
     \end{subfigure}
  \end{figure}

\end{frame}


\begin{frame}

\begin{textblock}{128}(0,10)
\begin{center}
 \large Probabilidad de ganar (anal\'itica)
\end{center}
\end{textblock}
\vspace{0.75cm}

\begin{equation*}
\begin{split}
  d_{ij} & \sim \iint \mathbb{I}(d_{ij}=p_i-p_j) N(p_i|s_i,\beta^2)N(p_j|s_j,\beta^2) \, dp_i \, dp_j \\
  \onslide<2->{&\overset{1}{=} \int N(d_{ij}+p_j|s_i,\beta^2) N(p_j|s_j,\beta^2) dp_j}\\
  \onslide<3->{&\overset{2}{=} \int N(p_j|s_i-d_{ij},\beta^2) N(p_j|s_j,\beta^2) dp_j} \\
  \onslide<4->{&\overset{*}{=} \int \underbrace{N(d_{ij}|s_i-s_j,2\beta^2)}_{\text{const.}} \underbrace{N(p_j|\mu_{-},\sigma^2_{-})dp_j}_{1}}
\end{split}
\end{equation*}
\vspace{0.25cm}
\onslide<5->{
\begin{equation}
d_{ij} \sim N(d_{ij}|s_i-s_j,2\beta^2)
\end{equation}}

\end{frame}

\begin{frame}
\vspace{-0.1cm}
\begin{equation}
P(p_i>p_j|s_i,s_j) = 1 - \Phi\left( \frac{0-(s_i - s_j)}{\sqrt{2}\beta} \right) =  \Phi\left( \frac{s_i - s_j}{\sqrt{2}\beta} \right) 
\end{equation}
\vspace{-0.1cm}
\begin{figure}[H]     
     \centering
     \begin{subfigure}[b]{0.65\textwidth}
       \includegraphics[page=1,width=\textwidth]{figures/probaOfWin} 
     \end{subfigure}
\end{figure}

\end{frame}

\subsection{Actualizaci\'on}
 
\begin{frame}
\begin{textblock}{128}(0,10)
\begin{center}
 \large Actualizaci\'on 
\end{center}
\end{textblock}
\vspace{0.75cm}

\begin{equation*}
\Delta = \underbrace{(2r_{ij}-1)}_{\hfrac{\text{\textbf{Direcci\'on} }}{\text{(Signo del resultado)}}} \, \underbrace{(1 - P(r_{ij}|s_i,s_j))}_{\hfrac{\text{\textbf{Magnitud}}}{\text{(Sorpresa del resultado)}}}
\end{equation*}

\vspace{0.25cm}

\begin{equation}
s_i^{\text{new}} = s_i^{\text{old}} + K \Delta
\end{equation}

\vspace{0.25cm}

\onslide<2->{
\begin{center}
\large Modelo de soluci\'on  
\end{center}
\vspace{-0.1cm}
\begin{mdframed}[backgroundcolor=black!20]
\begin{center}
 Estimador $\hfrac{\longrightarrow}{\longleftarrow}$ Predicci\'on
\end{center}
\end{mdframed}


\begin{itemize}
 \item[$\bullet$] El estimador hace a la predicci\'on
 \item[$\bullet$] La predicci\'on hace al estimador
\end{itemize}
}



\end{frame}

\subsection{Problemas}

\begin{frame}

{\large Debilidades del sistema Elo}

\begin{enumerate}
  \item La actualizaci\'on depende de un valor arbitrario, $K$
  \item No hay noci\'on de intervalo de confianza
  \item Solo sirve para competencias individuales
 \end{enumerate}

 
\end{frame}


\section{TrueSkill}
% 
% \begin{frame}
%  
% \begin{center}
%  \Large TrueSkill
% %\includegraphics[page=1,width=0.25\textwidth]{images/microsoft} 
% \end{center}
% 
% \begin{figure}[H]     
%      \centering
%      \begin{subfigure}[b]{0.3\textwidth}
%        \includegraphics[page=1,width=\textwidth]{images/ralfHerbrich} 
%      \caption*{Ralf Herbrich}
%      \end{subfigure}
%      \begin{subfigure}[b]{0.3\textwidth}
%        \includegraphics[page=1,width=\textwidth]{images/tomMinka} 
%      \caption*{Tom Minka}
%      \end{subfigure}
%      \begin{subfigure}[b]{0.3\textwidth}
%        \includegraphics[page=1,width=\textwidth]{images/thoreGraepel} 
%      \caption*{Thore Graepel}
%      \end{subfigure}
%      
%   \end{figure}
%  
% \end{frame}


\subsection{Modelo}

\begin{frame}

\begin{textblock}{128}(0,10)
\begin{center}
 \large Modelo TrueSkill
\end{center}
\end{textblock}
\vspace{0.5cm}

\begin{figure}[H]
  \scalebox{.85}{\tikz{ %
        
        \node[det, fill=black!10] (r) {$r_j$} ; %
        \node[const, below=of r, yshift=-0.8cm,xshift=-0.8cm] (c_1a) {\footnotesize  $o$:= ordenamiento};
        \node[latent, left=of r] (d) {$d_j$} ; %
        \node[latent, left=of d] (t) {$t_e$} ; %
        \node[latent, left=of t] (p) {$p_i$} ; %
        \node[latent, left=of p, yshift=-0.6cm] (s) {$s_i$} ; %
        \node[obs, left=of p, yshift=0.6cm] (beta) {$\beta$} ; %
        \node[obs, left=of s] (mu) {$\mu_i$} ; %
        \node[obs, left=of s, yshift=1.2cm] (sigma) {$\sigma_i$} ; %
        
        
        \edge {d} {r};
        \edge {t} {d};
        \edge {p} {t};
        \edge {s} {p};
        \edge {beta} {p};
	\edge {mu,sigma} {s};
	
        \plate {personas} {(p)(s)(beta)(mu)(sigma)} {$i \in A_e$}; %
        \plate {equipos} {(personas) (t)} {$  \text{$A$:= partici\'on de jugadores }  \ \ 0 < e \leq |A|$}; %
	\node[invisible, below=of d, yshift=-1cm,xshift=-0.5cm] (inv_below) {};
	\node[invisible, above=of r, yshift=0.6cm] (inv_above) {};
	\plate {comparaciones} {(d) (r) (inv_below) (inv_above)} {$0 < j < |A|$}
	
	\node[const, right= of r, xshift=1.2cm ,yshift=-2.1cm] (result-dist) {$r_j = d_j > 0$} ; %
	\node[const, above=of result-dist,yshift=0.3cm] (d-dist) {$d_j = t_{o_j} - t_{o_{j+1}}$};  %
	\node[const, above=of d-dist,yshift=0.3cm] (t-dist) {$t_e = \sum_{i\in A_e} p_i $} ; %
	\node[const, above=of t-dist,yshift=0.3cm] (p-dist) {$p_i \sim N(s_i,\beta^2)$} ; %
	\node[const, above=of p-dist,yshift=0.3cm] (s-dist) {$s_i \sim N(\mu_i,\sigma_i^2)$} ; %
              
        }
}
  %\caption{\small Modelo \texttt{Trueskill} sobre habilidad de jugadores que compiten entre si}
  \label{modelo_trueskill}
\end{figure}

\begin{equation}
 P(s|o,A) = \frac{P(o|s,A)P(s)}{P(o|A)}
\end{equation}


\end{frame}

 \subsection{Personas}
 
 \begin{frame}
 
  \begin{equation*}
   p_i \sim  N(s_i, \beta^2)= \int N(p_i|s_i,\beta^2) N(s_i|\mu_i,\sigma_i^2) ds_i 
  \end{equation*}

 
 \Wider[4em]{
 \onslide<2->{
\begin{figure}[H]     
     \centering
     \begin{subfigure}[b]{0.32\textwidth}
       \includegraphics[page=1,width=1\textwidth]{figures/paso_1_multiplicacion_normales} 
     \end{subfigure}
     \onslide<3->{
     \begin{subfigure}[b]{0.32\textwidth}
       \includegraphics[page=1,width=1\textwidth]{figures/paso_1_multiplicacion_normales_image} 
     \end{subfigure}}
     \onslide<4->{
     \begin{subfigure}[b]{0.32\textwidth}
       \includegraphics[page=1,width=\textwidth]{figures/paso_1_multiplicacion_normales_p} 
     \end{subfigure}}
  \end{figure}
}
}
  
  \begin{equation*}
  \begin{split}
\onslide<5->{&\overset{1}{=} \int N(s_i|p_i,\beta^2) N(s_i|\mu_i,\sigma_i^2) ds_i \\}
\onslide<6->{& \overset{*}{=} \int \underbrace{N(p_i|\mu_i,\beta^2 + \sigma_i^2)}_{\text{const.}} \underbrace{N(s_i | \mu_*,\sigma_{*}^2)ds_i}_{1}}
  \end{split}
 \end{equation*}

 \onslide<7->{ 
 \begin{equation}
  p_i \sim N(\mu_i,\beta^2 + \sigma_i^2)
 \end{equation}
 }

 
 \end{frame}

\subsection{Equipos}

\begin{frame}
 
\begin{equation*}
 t_e \sim \int \dots \int \mathbb{I}(t_e = \sum_{j \in A_e} p_j) \big( \prod_{i \in A_e} N(p_i|\mu_i,\beta^2 + \sigma_i^2) \big) d\vec{p}
\end{equation*}

 Equipo de 2
 \vspace{-0.5cm}
\begin{columns}
\hspace{-2cm}
\begin{column}{0.33\textwidth}
  
  \begin{figure}[H]     
      \centering
      \vspace{-.25cm}
      \begin{subfigure}[b]{1\textwidth}
	\includegraphics[page=1,width=1.05\textwidth]{figures/paso_3_suma_normales_image} 
      \end{subfigure}
    \end{figure}
  
\end{column}
\hspace{-2cm}
\begin{column}{0.66\textwidth}

  \footnotesize
  \begin{equation*}
  \begin{split}
  \onslide<2->{& = \iint \mathbb{I}(t_e = p_i + p_j) N(p_i|\mu_i,\beta^2 + \sigma_i^2)N(p_j|\mu_j,\beta^2 + \sigma_j^2) dp_idp_j \\}
  \onslide<3->{& \overset{1}{=} \int N(p_i|\mu_i,\beta^2 + \sigma_i^2) N(t_e - p_i|\mu_j,\beta^2 + \sigma_j^2) dp_i   \\}
  \onslide<4->{& \overset{2}{=} \int N(p_i|\mu_i,\beta^2 + \sigma_i^2) N(p_i|t_e - \mu_j,\beta^2 + \sigma_j^2) dp_i   \\}
  \onslide<5->{& \overset{*}{=} \int \underbrace{N(t_e|\mu_i+\mu_j,2\beta^2 + \sigma_i^2 + \sigma_j^2)}_{\text{const.}} \underbrace{N(p_i|\mu_{*},\sigma_{*}^2) dp_i}_{1} \\ }
  \end{split}
  \end{equation*} 
\end{column}
\end{columns}

\onslide<6->{
General (por inducci\'on)
\begin{equation}
 t_e \sim  N \Big( \sum_{i\in A_e } \mu_i, \sum_{i \in A_e} \beta^2 + \sigma_i^2 \Big)
\end{equation}
}

 \end{frame}


\subsection{Resultado}



\begin{frame}

\begin{textblock}{128}(0,10)
\begin{center}
 \large Probabilidad de ganar
\end{center}
\end{textblock}
\vspace{0.75cm}

Diferencia
\footnotesize
 \Wider[6em]{
 \begin{equation*}
 d_{ab} \sim \iint \mathbb{I}(d_{ab}=t_a -t_b)N\Big(t_a|\sum_{i\in A_a} \mu_i,\sum_{i\in A_a} \beta^2 + \sigma_i^2\Big)\,N\Big(t_b|\sum_{i\in A_b} \mu_i,\sum_{i\in A_b} \beta^2 + \sigma_i^2\Big) dt_adt_b 
\end{equation*}
}
\onslide<2->{
\begin{equation}
d_{ab} \sim N\Bigg( \underbrace{\sum_{i\in A_a} \mu_i - \sum_{i\in A_b} \mu_i }_{\text{Diferencia esperada} (\delta)} ,\underbrace{\sum_{i\in A_a\cup A_b} \beta^2 + \sigma_i^2}_{\text{Varianza total} (\vartheta)} \Bigg)
\end{equation}
}
\vspace{0.5cm}
\onslide<3->{
Resultado
\begin{equation}
 P(r_{ab}=1|A) = P(d_{ab} > 0 | A) = \Phi\left( \frac{\delta}{\sqrt{2\vartheta} } \right)
\end{equation}
}
\end{frame}

\subsection{Actualizaci\'on}

\begin{frame}
 
\begin{textblock}{128}(0,10)
\begin{center}
 \Large Posterior
\end{center}
\end{textblock}
\vspace{0.75cm}

\begin{mdframed}[backgroundcolor=black!20]
\centering \large
\vspace{0.1cm}
 ¿Alquien sabe c\'omo calcular la posterior?
 \vspace{0.1cm}
\end{mdframed}
\vspace{1.5cm}
\begin{equation*}
 P(s_i|o,A) = \frac{P(o|s_i,A)P(s_i)}{P(o|A)}
\end{equation*}
 
 
\end{frame}


\section{Sum-product algorithm}

\subsection{Factor graph}

\begin{frame}
 
\begin{textblock}{128}(0,9)
\begin{center}
 \large Sum-product algorithm
\end{center}
\end{textblock}
\vspace{1cm}

\begin{equation*}
\footnotesize
\begin{split} 
%&g: R_1 \times \dots \times R_n \longmapsto D \\ 
g(x_1,\dots,x_n) = \prod_{j \in J} f_j(X) 
\end{split}
\end{equation*}

Ejemplo
\begin{equation*}
g(x_1,x_2,x_3,x_4,x_5) = f_A(x_1)f_B(x_2)f_C(x_1,x_2,x_3)f_D(x_3,x_4)f_E(x_3,x_5)
\end{equation*}
\vspace{-0.5cm}
\begin{figure}[H]
\centering
  \caption*{\scriptsize Factor graph}
  \vspace{-0.1cm}
  \scalebox{.74}{\tikz{ %
        
        \node[latent] (x1) {$x_1$} ; %
        \node[factor, below=of x1] (fA) {} ; 
        \node[const, right=of fA] () {$f_A$}; %
        \node[latent, right=of x1] (x2) {$x_2$} ; %
        \node[factor, below=of x2] (fB) {} ;
        \node[const, right=of fB] () {$f_B$}; %
        \node[latent, right=of x2] (x3) {$x_3$} ; %
        \node[factor, below=of x3] (fC) {} ;
        \node[const, right=of fC] () {$f_C$}; %
        \node[latent, right=of x3] (x4) {$x_4$} ; %
        \node[factor, below=of x4] (fD) {} ;
        \node[const, right=of fD] () {$f_D$}; %
        \node[latent, right=of x4] (x5) {$x_5$} ; %
        \node[factor, below=of x5] (fE) {} ;
        \node[const, right=of fE] () {$f_E$}; %
     
        \edge[-] {x1} {fA,fC};
        \edge[-] {x2} {fB,fC};
	\edge[-] {x3} {fC,fD,fE};
	\edge[-] {x4} {fD};
	\edge[-] {x5} {fE};
	
	
        }}
\end{figure}

\pause


\begin{framed}
Un factor graph sin ciclos codifica:
\begin{itemize}
 \item[$\bullet$] La facotirzaci\'on de la funci\'on $g(x_1,\dots,x_n)$
 \item[$\bullet$] \textbf{Las operaciones para computar sus marginales} $g_i(x_i)$
\end{itemize} 
\end{framed}

 
\end{frame}

\subsection{Marginales}

\begin{frame}

\begin{center}
 C\'omputo de marginales
\end{center}


\begin{equation}
g_i(x_i) = \prod_{h \in n(x_i)} m_{h \rightarrow x_i}
\end{equation} 

\begin{center}
-----------------------------------------------------------------------
\end{center}


\begin{equation}\label{m_v_f} 
m_{x \rightarrow f}(x) = \prod_{h \in n(x) \setminus \{f\} } m_{h \rightarrow x}(x)
\end{equation}
\begin{equation}\label{m_f_v}  
m_{f \rightarrow x}(x) = \sum_{X\setminus \{x\} } \left( f(X) \prod_{h \in n(f) \setminus \{x\} } m_{h \rightarrow f}(h) \right)
\end{equation}

Donde,
\begin{description}
 \item[$m_{x \rightarrow f}(x)$ :] Mensaje enviado por el nodo variable $x$ al nodo factor $f$
 \item[$m_{f \rightarrow x}(x)$ :] Mensaje enviado por un nodo factor $f$ a un nodo variable $x$.
 \item[$n(v)$ :] Conjunto de nodos vecinos a $v$.
\end{description}


\end{frame}


\subsection{Ejemplo}

\begin{frame}

\begin{equation*}
g(x_1,x_2,x_3,x_4,x_5) = f_A(x_1)f_B(x_2)f_C(x_1,x_2,x_3)f_D(x_3,x_4)f_E(x_3,x_5)
\end{equation*}

\vspace{0.1cm}

\begin{center}
Marginales 
\end{center}
\vspace{-0.5cm}
\begin{figure}[H]
\centering
  \begin{subfigure}[b]{0.48\textwidth} 
  \centering
    \scalebox{0.6}{\tikz{ %
        
        \node[latent] (x1) {$x_1$} ; %
        
        \node[factor, below=of x1,xshift=-1cm] (fA) {} ; 
        \node[const, left=of fA] () {$f_A$}; %              
        \node[factor, below=of x1,xshift=1cm] (fC) {} ;
        \node[const, right=of fC] () {$f_C$}; %
        %\node[det, left=of fA,xshift=-0.6cm] () {$\sum_{\setminus x_1}$}; %
        
        \node[latent, below=of fC,xshift=1cm] (x3) {$x_3$} ; %
        \node[latent, below=of fC,xshift=-1cm] (x2) {$x_2$} ; %
        
        \node[factor, below=of x2,xshift=-1cm] (fB) {} ;
        \node[const, left=of fB] () {$f_B$}; %
        \node[factor, below=of x3,xshift=-1cm] (fD) {} ;
        \node[const, left=of fD] () {$f_D$}; %
        \node[factor, below=of x3, xshift=1cm] (fE) {} ;
        \node[const, right=of fE] () {$f_E$}; %
        %\node[const, left=of fB,xshift=-0.6cm] () {$\sum_{\setminus x_3}$}; %
           
        \node[latent, below=of fD] (x4) {$x_4$} ; %
        \node[latent, below=of fE] (x5) {$x_5$} ; %
     
	\edge[-] {x1} {fA,fC};
        \edge[-] {x2} {fB,fC};
	\edge[-] {x3} {fC,fD,fE};
	\edge[-] {x4} {fD};
	\edge[-] {x5} {fE};
	
	
        }}
    \caption*{$g_1(x_1)$}
  \end{subfigure} 
\ \ \
  \begin{subfigure}[b]{0.48\textwidth} 
  \centering
    \scalebox{0.6}{\tikz{ %
        
        \node[latent] (x3) {$x_3$} ; %
        
        \node[factor, below=of x3,xshift=-1.5cm] (fC) {} ;
        \node[const, left=of fC] () {$f_C$}; %
        \node[factor, below=of x3] (fD) {} ;
        \node[const, left=of fD] () {$f_D$}; %
        \node[factor, below=of x3, xshift=1.5cm] (fE) {} ;
        \node[const, right=of fE] () {$f_E$}; %
        
        \node[latent, below=of fC,xshift=-1.5cm] (x1) {$x_1$} ; %
        \node[latent, below=of fC] (x2) {$x_2$} ; %
        \node[latent, below=of fD] (x4) {$x_4$} ; %
        \node[latent, below=of fE] (x5) {$x_5$} ; %
     
        \node[factor, below=of x1] (fA) {} ; 
        \node[const, left=of fA] () {$f_A$}; %              
        \node[factor, below=of x2] (fB) {} ;
        \node[const, left=of fB] () {$f_B$}; %     
        
	\edge[-] {x1} {fA,fC};
        \edge[-] {x2} {fB,fC};
	\edge[-] {x3} {fC,fD,fE};
	\edge[-] {x4} {fD};
	\edge[-] {x5} {fE};
	
	
        }}
   \caption*{$g_3(x_3)$}
  \end{subfigure} 
\end{figure}
% 
% {\footnotesize
% \begin{equation*}
%  g_1(x_1) = f_A(x_1) \left(\sum_{\setminus x_1} f_B(x_2) f_C(x_1,x_2,x_3) \left( \sum_{\setminus x_3} f_D(x_3,x_4) \right) \left( \sum_{\setminus x_3} f_E(x_3,x_5) \right) \right)
% \end{equation*}
% 
% \begin{equation*}
%  g_3(x_3) =  \left( \sum_{\setminus x_3} f_A(x_1)f_B(x_2)f_C(x_1,x_2,x_3)  \right) \left( \sum_{\setminus x_3} f_D(x_3,x_4) \right) \left( \sum_{\setminus x_3} f_E(x_3,x_5) \right) 
% \end{equation*}
% }
\end{frame}

\section{TrueSkill}

\subsection{Factor graph}

\begin{frame}

\Wider[4em]{
\begin{figure}[H]
\centering
  \begin{subfigure}[b]{1\textwidth} 
  \centering
    \scalebox{0.8}{\tikz{ %
        
      
        \node[factor] (fr) {} ;
        \node[const, above=of fr] (nfr) {$f_r$}; %
	\node[const, above=of nfr] (dfr) {\large $\mathbb{I}(d_j>0)$}; %
        \node[latent, left=of fr] (d) {$d_j$} ; %
        \node[factor, left=of d] (fd) {} ;
        \node[const, above=of fd] (nfd) {$f_d$}; %
        \node[const, above=of nfd] (dfd) {\large $\mathbb{I}(d_j=t_{o_j} - t_{o_{j+1}})$}; %
        \node[const, below=of d,yshift=-0.15cm] (j) {\footnotesize $o:=$ ordenamiento observado};
        
        \node[latent, left=of fd,xshift=-1.3cm] (t) {$t_e$} ; %
        \node[factor, left=of t] (ft) {} ;
        \node[const, above=of ft] (nft) {$f_t$}; %
        \node[const, above=of nft,xshift=0.5cm] (dft) {\large $\mathbb{I}(t_e = \sum_{i \in A_e} p_i)$}; %
        
        \node[latent, left=of ft, xshift=-0.3cm] (p) {$p_i$} ; %
        \node[factor, left=of p] (fp) {} ;
        \node[const, above=of fp] (nfp) {$f_p$}; %
        \node[const, above=of nfp] (dfp) {\large $N(p_i|s_i,\beta^2)$}; %
 
        \node[latent, left=of fp] (s) {$s_i$} ; %
        \node[factor, left=of s] (fs) {} ;
        \node[const, above=of fs] (nfs) {$f_s$}; %
        \node[const, above=of nfs] (dfs) {\large $N(s_i|\mu_i,\sigma^2)$}; %
         
        \edge[-] {d} {fr};
	\edge[-] {fd} {d};
        \edge[-] {fd} {t};
        \edge[-] {t} {ft};
        \edge[-] {ft} {p};
        \edge[-] {p} {fp};
        \edge[-] {fp} {s};
        \edge[-] {s} {fs};
	
        \plate {personas} {(p)(s)(fs)(nfs)(dfp)(dfs)} {$i \in A_e$}; %
        \node[invisible, below=of ft, yshift=-0.6cm] (inv_below_e) {};
	\node[invisible, above=of ft, yshift=1.1cm] (inv_above_e) {};
	\plate {equipos} {(personas) (t)(ft)(dft) (inv_above_e) (inv_below_e)} {$A := \text{partici\'on de jugadores} \hspace{2.5cm} 0 < e \leq |A|$}; %
	\node[invisible, below=of fr, yshift=-0.675cm,size=0pt] (inv_below) {};
	\node[invisible, above=of fr, yshift=1.1cm,size=0pt] (inv_above) {};
	\plate {comparaciones} {(fd) (dfd) (d) (fr) (dfr) (inv_below) (inv_above)} {$0 < j < |A|$}
	
	
	%\node[const, right= of r, xshift=1.2cm ,yshift=-2.1cm] (result-dist) {$r_{ab} \sim B\left(\Phi\left(\frac{\mu_a - \mu_b}{\sqrt{\beta_a^2+\beta_b^2}}\right)\right)$} ; %
	      
        }
}
  \end{subfigure} 
\end{figure}
}

\pause
\vspace{0.75cm}

\Wider[4em]{
\begin{figure}[H]
\centering
  \begin{subfigure}[b]{1\textwidth} 
  \centering
    \scalebox{0.75}{\tikz{ %
        
      
        \node[factor] (fr) {} ;
        \node[const, right=of fr] (nfr) {$f_{r_1}$}; %
	
	\node[latent, above=of fr] (d) {$d_1$} ; %
        \node[factor, above=of d] (fd) {} ;
        \node[const, right=of fd] (nfd) {$f_{d_1}$}; %
	
        
        \node[latent, above=of fd,xshift=-0.75cm] (ta) {$t_a$} ; %
        \node[factor, left=of ta] (fta) {} ;
        \node[const, above=of fta] (nfta) {$f_{t_a}$}; %
        
        
        
        \node[latent, left=of fta,yshift=1cm] (p1) {$p_1$} ; %
        \node[factor, left=of p1] (fp1) {} ;
        \node[const, above=of fp1] (nfp1) {$f_{p_1}$}; %
        
        \node[latent, left=of fp1] (s1) {$s_1$} ; %
        \node[factor, left=of s1] (fs1) {} ;
	\node[const, above=of fs1] (nfs1) {$f_{s_1}$}; %
     
        \node[latent, left=of fta,yshift=-1cm] (p2) {$p_2$} ; %
        \node[factor, left=of p2] (fp2) {} ;
        \node[const, above=of fp2] (nfp2) {$f_{p_2}$}; %
        
        \node[latent, left=of fp2] (s2) {$s_2$} ; %
        \node[factor, left=of s2] (fs2) {} ;
	\node[const, above=of fs2] (nfs2) {$f_{s_2}$}; %
        
            
        \node[latent, above=of fd,xshift=0.75cm] (tb) {$t_b$} ; %
        \node[factor, right=of tb] (ftb) {} ;
        \node[const, above=of ftb] (nftb) {$f_{t_b}$}; %
        
        \node[latent, right=of ftb,yshift=1cm] (p3) {$p_3$} ; %
        \node[factor, right=of p3] (fp3) {} ;
        \node[const, above=of fp3] (nfp3) {$f_{p_3}$}; %
        
        \node[latent, right=of fp3] (s3) {$s_3$} ; %
        \node[factor, right=of s3] (fs3) {} ;
	\node[const, above=of fs3] (nfs3) {$f_{s_3}$}; %
     
        \node[latent, right=of ftb,yshift=-1cm] (p4) {$p_4$} ; %
        \node[factor, right=of p4] (fp4) {} ;
        \node[const, above=of fp4] (nfp4) {$f_{p_4}$}; %
        
        \node[latent, right=of fp4] (s4) {$s_4$} ; %
        \node[factor, right=of s4] (fs4) {} ;
	\node[const, above=of fs4] (nfs4) {$f_{s_4}$}; %
     
        \edge[-] {fr} {d};
	\edge[-] {d} {fd};
	
        \edge[-] {fd} {ta};
        \edge[-] {ta} {fta};
        \edge[-] {fta} {p1};
        \edge[-] {p1} {fp1};
        \edge[-] {fp1} {s1};
        \edge[-] {s1} {fs1};
        \edge[-] {fta} {p2};
        \edge[-] {p2} {fp2};
        \edge[-] {fp2} {s2};
        \edge[-] {s2} {fs2};
        	
	\edge[-] {fd} {tb};
        \edge[-] {tb} {ftb};
        \edge[-] {ftb} {p3};
        \edge[-] {p3} {fp3};
        \edge[-] {fp3} {s3};
        \edge[-] {s3} {fs3};
        \edge[-] {ftb} {p4};
        \edge[-] {p4} {fp4};
        \edge[-] {fp4} {s4};
        \edge[-] {s4} {fs4};
%         
% 	
% 	\node[const, below=of nfr,xshift=7cm,yshift=-1cm] (dfr) { $fr_k = \mathbb{I}(d_k>0)$}; %
% 	\node[const, left=of dfr,xshift=-0.5cm] (dfd) {$fd_k = \mathbb{I}(d_k=t_{(k)} - t_{(k+1)})$}; %
% 	\node[const, left=of dfd,xshift=-0.5cm] (dft) {$ft_e = \mathbb{I}(t_e = \sum_{i \in A_e} p_i)$}; %
%         \node[const, left=of dft,xshift=-0.5cm] (dfp) {$fp_i = N(p_i;s_i,\beta^2)$}; %
%         \node[const, left=of dfp,xshift=-0.5cm] (dfs) {$fs_i = N(s_i;\mu_i,\sigma^2)$}; %
%  
	
	%\node[const, right= of r, xshift=1.2cm ,yshift=-2.1cm] (result-dist) {$r_{ab} \sim B\left(\Phi\left(\frac{\mu_a - \mu_b}{\sqrt{\beta_a^2+\beta_b^2}}\right)\right)$} ; %
	      
        }
}
  \end{subfigure} 
\end{figure}
}

\end{frame}

\subsection{Pasaje de mensajes}

\begin{frame}

\begin{textblock}{128}(0,10)
\begin{center}
 \large $\bm{m_{f_r \rightarrow d}(d)}$
 \vspace{0.3cm}
 \scriptsize \transparent{0.4}
 $$m_{f \rightarrow x}(x) = \int\dots\int f(X) \prod_{h \in n(f) \setminus \{x\} } m_{h \rightarrow f}(h) \ d_{X\setminus \{x\} }$$
\end{center}
\end{textblock}
\vspace{1cm}

\begin{equation}
m_{f_r \rightarrow d_k}(d_k) = \mathbb{I}(d_k > 0)
\end{equation}
 
\end{frame}


\begin{frame}
\begin{textblock}{128}(0,10)
\begin{center}
 \large $\bm{m_{d \rightarrow f_{t}}(d)}$
 \vspace{0.3cm}
 \scriptsize \transparent{0.4}
 $$m_{f \rightarrow x}(x) = \prod_{h \in n(x) \setminus \{f\} } m_{h \rightarrow x}(x) \ $$
\end{center}
\end{textblock}
\vspace{1cm}

\begin{equation}
m_{d_k \rightarrow f_{t_a}}(d_k) = \mathbb{I}(d_k > 0)
\end{equation}

\end{frame}

\begin{frame}
\begin{textblock}{128}(0,10)
\begin{center}
 \large $\bm{m_{f_d \rightarrow t}(t)}$
\end{center}
\end{textblock}
\vspace{0.75cm}

\Wider[4em]{
\begin{equation*}
\begin{split}
m_{f_{d_1} \rightarrow t_a}(t_a) &= \iint \mathbb{I}(d_1 = t_a - t_b) \mathbb{I}(d_1 > 0) N\Big(t_b | \sum_{i \in A_b} \mu_i , \sum_{i \in A_b} \beta^2 + \sigma_i^2 \Big) dt_b dd_1 \\
\onslide<2->{&\overset{1}{=}\int \mathbb{I}( t_a > t_b)  N\Big(t_b | \sum_{i \in A_b} \mu_i , \sum_{i \in A_b} \beta^2 + \sigma_i^2 \Big)}
\end{split}
\end{equation*}

}

\Wider{


\begin{columns}
\begin{column}{0.4\textwidth}
\onslide<3->{  
  \begin{figure}[H]
  \centering
    \begin{subfigure}[b]{1\textwidth} 
    \centering
      \includegraphics[width=\textwidth]{figures/m_d_ta} 
    \end{subfigure} 
  \end{figure}
}
\end{column}
\hspace{-1cm}
\begin{column}{0.66\textwidth}
\onslide<4->{  
  \begin{equation}
    m_{f_{d_1} \rightarrow t_a}(t_a) \overset{2}{=} \Phi \Big(t_a| \sum_{i \in A_b} \mu_i , \sum_{i \in A_b} \beta^2 + \sigma_i^2 \Big)
  \end{equation} 
}
\end{column}
\end{columns}

}
 
\end{frame}

\begin{frame}
\begin{textblock}{128}(0,10)
\begin{center}
 \large $\bm{m_{f_t \rightarrow p}(p)}$
\end{center}
\end{textblock}
\vspace{0.75cm}

\Wider[6em]{
\footnotesize
\begin{equation*}
 \begin{split}
  m_{f_{t_a} \rightarrow p_1}(p_1) =& \iint \mathbb{I}( t_a = p_1 + p_2) \Phi \Big(t_a| \sum_{i \in A_b} \mu_i , \sum_{i \in A_b} \beta^2 + \sigma_i^2 \Big) \, N(p_2| \mu_2, \beta^2 + \sigma_2^2 ) \, dt_a dp_2 \\
  \onslide<2->{=&\int \Phi \Big(p1+p_2| \sum_{i \in A_b} \mu_i , \sum_{i \in A_b} \beta^2 + \sigma_i^2 \Big) \, N(p_2| \mu_2, \beta^2 + \sigma_2^2 ) \, dp_2}
  \end{split}
\end{equation*}
}

\end{frame}


\begin{frame}
\begin{textblock}{128}(0,10)
\begin{center}
 \large $\bm{m_{f_t \rightarrow p}(p)}$
\end{center}
\end{textblock}
\vspace{0.75cm}

\Wider[4em]{
\footnotesize
\begin{equation*}
\begin{split}
\frac{\partial m_{f_{t_a} \rightarrow p_1}(p_1)}{\partial p_1} &\overset{1}{=} \frac{\partial}{\partial p_1} \int  \Phi \Big(p_1 + p_2| \sum_{i \in A_b} \mu_i , \sum_{i \in A_b} \beta^2 + \sigma_i^2 \Big) N(p_2| \mu_2, \beta^2 + \sigma_2^2 ) \, dp_2 \\
\onslide<2->{&\overset{2}{=} \int  N(p_2| \mu_2, \beta^2 + \sigma_2^2 ) \, \frac{\partial}{\partial p_1}\Phi \Big(p_1| \sum_{i \in A_b} \mu_i - p_2 , \sum_{i \in A_b} \beta^2 + \sigma_i^2 \Big)  \, dp_2   \\}
\onslide<3->{&\overset{3}{=} \int  N(p_2| \mu_2, \beta^2 + \sigma_2^2 ) \, N\Big(p_1| \sum_{i \in A_b} \mu_i - p_2 , \sum_{i \in A_b} \beta^2 + \sigma_i^2 \Big)  \, dp_2  \\}
\onslide<4->{&\overset{4}{=} \int  N(p_2| \mu_2, \beta^2 + \sigma_2^2 ) \, N\Big(p_2| \sum_{i \in A_b} \mu_i - p_1 , \sum_{i \in A_b} \beta^2 + \sigma_i^2 \Big)  \, dp_2  \\}
\onslide<5->{&\overset{*}{=} N\Big(p_1| \sum_{i \in A_b} \mu_i - \mu_2, \beta^2 + \sigma_2^2 + \sum_{i \in A_b} \beta^2 + \sigma_i^2 \Big) }
\end{split}
\end{equation*}
\normalsize
\onslide<6->{
\begin{equation}
 m_{f_{t_a} \rightarrow p_1}(p_1) = \Phi\Big(p_1| \mu_1-\delta, \vartheta -  \beta^2 + \sigma_1^2 \Big) 
\end{equation}
}

}

 
\end{frame}


\begin{frame}
\begin{textblock}{128}(0,10)
\begin{center}
 \large $\bm{m_{f_p \rightarrow s}(s)}$
\end{center}
\end{textblock}
\vspace{0.75cm}

\Wider[4em]{
\small
\begin{equation*}
\begin{split}
m_{f_{p_1} \rightarrow s_1}(s_1) =& \int N(p_1| s_1, \beta^2) \Phi\Big(p_1| \mu_1-\delta, \vartheta -  \beta^2 + \sigma_1^2 \Big)   dp_1 \\[0.3cm]
\onslide<2->{\frac{\partial m_{f_{p_1} \rightarrow s_1}(s_1)}{\partial \mu_1}  &\overset{1}{=} \int N(p_1| s_1, \beta^2) N\Big(p_1| \mu_1-\delta, \vartheta -  \beta^2 + \sigma_1^2 \Big)   dp_1 \\}
\onslide<3->{&\overset{*}{=} N\Big(s_1| \mu_1-\delta, \vartheta - \sigma_1^2 \Big) }
\end{split}
\end{equation*}
\vspace{0.25cm}
\normalsize
\onslide<4->{
\begin{equation}
\begin{split}
m_{f_{p_1} \rightarrow s_1}(s_1) &\overset{2}{=} \Phi\Big(s_1| \mu_1-\delta, \vartheta - \sigma_1^2 \Big) \\
\onslide<5->{&\overset{3}{=} \Phi\Big(\underbrace{\delta-\mu_1+s_1}_{\hfrac{\text{Diferencia}}{\text{parametrizada} } \ \delta(s_i)}| 0, \vartheta - \sigma_1^2 \Big)}
\end{split}
\end{equation}
}
}
 
\end{frame}

\subsection{Posterior}

\begin{frame}
\begin{textblock}{128}(0,10)
\begin{center}
 \large Posterior
\end{center}
\end{textblock}
\vspace{1cm}
 
\Wider[4em]{ 
\begin{equation}
 P(s_i \mid \bm{o}, A) \propto 
 \begin{cases}
  N(s_i \, | \, \mu_i, \sigma_i^2) \, \Phi(\delta(s_i) \, | \, 0, \vartheta - \sigma_i^2) & \text{Si gana} \\
  N(s_i \, | \, \mu_i, \sigma_i^2 ) \, (1 - \Phi(\delta(s_i) \, | \, 0, \vartheta - \sigma_i^2)) & \text{Si pierde}
 \end{cases}
\end{equation}
\pause
\vspace{-0.5cm}

  \begin{figure}[H]     
     \centering
     \begin{subfigure}[b]{0.49\textwidth}
       \includegraphics[page=2,width=\textwidth]{figures/posterior_ganador} 
     \end{subfigure}
     \begin{subfigure}[b]{0.49\textwidth}
       \includegraphics[page=2,width=\textwidth]{figures/posterior_perdedor} 
     \end{subfigure}
   \end{figure} 
}

\end{frame}

\subsection{Trabajo futuro}

\begin{frame}
 \begin{textblock}{128}(0,10)
\begin{center}
 \large Trabajo futuro
\end{center}
\end{textblock}
\vspace{1cm}


  {\normalsize Debilidades de TrueSkill}
 

 \begin{enumerate}
 
  \item El modelo $t \sim \sum_{i \in A_e} p_i$ genera sobrestimaci\'on por sinergia.
  \begin{equation}
   \frac{\text{Sinergia total}}{\text{Tamaño del equipo}} \text{Proporci\'on de partidas jugadas en equipo}
  \end{equation}

  \item Resultado ganador/perdedor como \'uncio dato observable
  
  \item Posterior aproximada
  
 \end{enumerate}
 
\end{frame}

\section{Gracias}

\begin{frame}
 
 \begin{center}
  \Large Gracias
 \end{center}

 \begin{figure}[H]     
     \centering
     \begin{subfigure}[b]{0.45\textwidth}
       \includegraphics[width=\textwidth]{images/pachacuteckoricancha.jpg} 
     \end{subfigure}
   \end{figure} 
  
\end{frame}


\end{document}



